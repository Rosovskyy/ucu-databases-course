\documentclass[12pt]{article}
\usepackage[utf8]{inputenc}
\usepackage[english]{babel}
\usepackage{hyperref}
\usepackage{indentfirst}

\hypersetup{ colorlinks, citecolor=green, filecolor=black, linkcolor=blue, urlcolor=blue }

\newcommand{\code}[1]{\texttt{#1}}

\begin{document}

\title{Practical Assignment 1: Basic SQL}
\author{Anatolii Stehnii\\Data architecture}
\maketitle

\section*{Intoduction}
The goal of this assignment is to give you an understanding of basic SQL operations: schema creation, data insert, update and delete. You will build a book's example of the \code{University} database with your own data.

\section*{Conventions}

This assignment will be tested by automated functional tests, therefore all your  actions in the database (table creation, user creation, insert, update) should be reflected in the corresponding \code{.sql} files. I recommend you to use the same environment (MySQL 5.7) when doing this assignment to ensure the compatibility of your results with the testing environment.

\section*{Recommendations}

Each practical assignment could contain some information not covered in lectures. But this should not stop you from its fulfillment. I encourage you to use Google to fill gaps in your practical knowledge\footnote{rtfm}. However, to reduce ambiguity between different information sources, I have added a list of recommended links with hints for each unknown point at the end of the assignment. It is no shame to use it, but first try to find information by yourself.

If you have any issues with this assignment, please reach me or course assistant Mykhailo Poliakov. Feel free to use the course Slack channel to ask questions and help your colleagues. If you have any new ideas or improvements, please to send pull-request to repository \href{https://github.com/tsdaemon/ucu-databases-course}{ucu-databases-course}. Your colleagues will be grateful if you make these instructions better.

\section*{Instructions}
\subsection*{Database creation}

\begin{enumerate}
\item Connect to your \code{MySQL} instance as \code{root} user.
\begin{verbatim}
mysql -u root -p
\end{verbatim}
\item Create database \code{university}.
\item Create a new user with all privileges on database \code{university}. Set user host to \code{\%}
\item Save your commands in \code{./db.sql}.
\item Finish root session.
\end{enumerate}

\subsection*{Schema and sample data}

\begin{enumerate}
\item Open spreadsheet with sample data \href{https://docs.google.com/spreadsheets/d/1fnpOlVMEQZJb3E07pDOHItz0ckAT2glQu6Vu9-_axvo/edit?usp=sharing}{here} and export it as CSV to your working folder as  \code{./students.csv}.
\item Connect to \code{MySQL} as the previously created user.
\item Create a table with the same fields (except you also need an ID) as in the sample data:

\begin{tabular}{| l | l | }
  \hline
  \textbf{name} & \textbf{type} \\
  \hline
  id & integer autoincremental primary key \\
  first\_name & text field up to 255 symbols, can not be empty \\
  last\_name & text field up to 255 symbols, can not be empty \\
  cohort & text field up to 3 symbols, can not be empty \\
  dormitory & boolean value, can not be empty \\
  pet  & text field up to 255 symbols, can be empty \\
  assignment1  & float, can be empty \\
  assignment2  & float, can be empty \\
  \hline
\end{tabular}

\item Load CSV file with sample data into the table.
\item Store your commands in \code{./table.sql}.

\end{enumerate}

\subsection*{Selecting data}

\begin{enumerate}
\item Select all students from your cohort, order by first name. Store this select as \code{select1.sql}.

\item Select all students from your cohort, order by first name and last name  in reverse to alphabetical. Store your code as \code{select1.sql}.

\item Select all students which the same first letter in a name as you. Order by the result of assignment 1 mark descending. Store your code as \code{select3.sql}.

\item Select first name, last name and average mark for students not from your cohort. Average mark is a mean value for assignment 1 mark and assignment 2 marks. Store your code as \code{select4.sql}.

\end{enumerate}

\subsection*{Inserting data}

\begin{enumerate}

\item You have a new student Vasyl Vasylev in cohort CS1, who lives in the dormitory and have a cat. Add information about him in the table.

\item Think up one more new student. Add information about him in the table.

\item Store your code as \code{insert.sql}.

\end{enumerate}

\subsection*{Updating data}

\begin{enumerate}

\item Update both your marks to 10.

\item Update marks for your friend to 9.5.

\item Correct dormitory field for your row. If it is currently correct, change it to the opposite.

\item Update your pet status. If it is currently correct, make it empty. If you have no pet, update it to 'None'.

\item Find your friend in the table and repeat steps 3 and 4 for him.

\item Store your code as \code{update.sql}.

\end{enumerate}

\subsection*{Deleting data}

\begin{enumerate}

\item Find somebody you want to expel (not literally) and delete his record.

\item Store your code as \code{delete.sql}.

\end{enumerate}

\subsection*{Changing schema}

\begin{enumerate}

\item Add new text field transport with max length 255 to your table. This field should have a default value \code{'empty'}.

\item Specify which transport use you and your friend.

\item Store your code as \code{transport.sql}.

\end{enumerate}

\section*{Submission}

To submit your solution, pack all code files in a single \textbf{zip} archive. Submit it on assignment page in CMS. After a successful solution upload, you will get a link to the CircleCI build page, which will show you the progress of your solution test.

\uppercase{\textbf{Please test your solution carefully before submission!}}\footnote{I am very aggressive when I waste my time on untested submissions.}

I give you credit for \textbf{one} submission fail. This means, if your test fails, you can check test response, correct your solution, and resubmit it. All the following errors will be penalized.

If you have any issues or errors during submission, please contact me or Mykhailo Poliakov.

\clearpage

\section*{Hints}
How to \dots
\begin{itemize}
\item \dots \href{https://dev.mysql.com/doc/refman/5.7/en/creating-database.html}{create database}.
\item \dots \href{https://www.digitalocean.com/community/tutorials/mysql-ru}{create user}.
\item \dots \href{https://dev.mysql.com/doc/refman/5.5/en/creating-tables.html}{create table}.
\item \dots \href{https://dev.mysql.com/doc/refman/5.7/en/example-auto-increment.html}{create identity column}.
\item \dots \href{https://dev.mysql.com/doc/refman/5.7/en/load-data.html}{import CSV into SQL}.

\end{itemize}

\end{document}
