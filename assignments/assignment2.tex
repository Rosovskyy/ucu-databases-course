\documentclass[12pt]{article}
\usepackage[utf8]{inputenc}
\usepackage{hyperref}
\usepackage{indentfirst}

\hypersetup{ colorlinks, citecolor=green, filecolor=black, linkcolor=blue, urlcolor=blue }

\newcommand{\code}[1]{\texttt{#1}}

\begin{document}

\title{Practical assignment 2: Associations}
\author{Anatolii Stehnii\\Intro to databases}
\maketitle

\section*{Description}

Second assignment purpose is to add a second table \code{student} into our application and create association between \code{lesson\_signal} and \code{student}. Also, this time you will use ORM to map SQL results to your application classes.

\section*{Conventions}

This assignment will be tested by automated functional tests, therefore all your manual actions in database (table creation, user creation etc) should be reflected in corresponding \code{.sql} files. I strictly advise you to use environment setup (DotNetCore2.0+MySQL) described in assignment 0 to ensure compatibility of your results with testing and production environments.

All file names are specified relatively to your \code{ucubot} repository location.

Note that for C\# accepted convention is to use \code{CamelCase} identifiers, while in MySQL names are usually \code{snake\_case}. Therefore a table \code{bot\_signal} corresponds to a class and an UML diagram \code{BotSignal}.

\section*{Recommendations}

Each practical assignment could contain some information not covered in lectures. But this should not stop you from its fulfillment. I encourage you to use Google to fill gaps in your practical knowledge\footnote{rtfm}. However, to reduce ambiguity between different information sources, I have added list of recommended links with hints for each unknown point in the end of the assignment. It is no shame to use it, but first try to find information by yourself.

\code{master} branch of repository \code{ucubot} contains working code of the application. But hold the temptation to watch it or copy the code. All your work should be performed in branch \code{learning} only by yourself without code copying. Please do not share your code with the colleagues or in the Slack channel.

If you have any issues with this assignment, please reach me in Slack @Anatolii Stehnii or send an email to \href{mailto:stehnii@ucu.edu.ua}{stehnii@ucu.edu.ua}. Feel free to use course Slack channel to ask questions and help your colleagues. If you have any new ideas or improvements, feel free to send pull-request to repository \href{https://github.com/tsdaemon/ucu-databases-course}{ucu-databases-course}. Your colleagues will be grateful if you make this instructions better.

\section*{Instructions}

\subsection*{New table}

\begin{enumerate}
\item Connect to \code{MySQL}.
\item Create a table \code{student} according to a following structure:
\begin{verbatim}
class Student {
    Id: int,
    FirstName: string,
    LastName: string,
    UserId: string
}
\end{verbatim}
\item Save all SQL code in \break \code{./ucubot/Scritps/student.sql}.
\end{enumerate}

\subsection*{Changes in lesson\_signal}

\begin{enumerate}

\item Connect to \code{MySQL}.
\item Remove column \code{user\_id} from table \code{lesson\_signal}.
\item Add foreign key column \code{student\_id} with reference to table \code{student}. Set restrict referential policy for UPDATE and DELETE actions.

\end{enumerate}

NOTE: If you have records in table \code{lesson\_signal}, adding foreign key column may result in error because table \code{student} yet have no records to be referenced. Consider two options as a solution:

\begin{itemize}

\item Remove all records from the table \code{lesson\_signal}

\dots or \dots

\item Allow null values in the field \code{student\_id}.

\end{itemize}


\clearpage

\section*{Hints}
How to \dots
\begin{itemize}
\item \dots \href{https://dev.mysql.com/doc/refman/5.6/en/create-table-foreign-keys.html}{create FOREIGN KEY}.
\end{itemize}

\end{document}
